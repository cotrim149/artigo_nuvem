\section{Ambiente}

No artigo analisado foi utilizado a ferramenta CloudSim e sua extensão CloudSim-DVFS. O CloudSim é
um simulador de nuvens computacionais, que pode ser utilizado para geração de uma nuvem privada.
No que se refere a nuvem privada, sua infraestrutura opera exclusivamente para uma única organização ou usuário,
seja ela gerida internamente ou por um terceiro, e hospedada internamente ou externamente.

O CloudSim-DVFS é uma extensão do simulador CloudSim que permite simulações energéticas tanto de uma
carga de trabalho quanto de workflows, além de fornecer uma nova versão da simulação da técnica
de dimensionamento dinâmico de voltagem e frequência (\textit{Dynamic voltage and frequency scaling – DVFS}).
Essa técnica permite mudar dinamicamente a voltagem e a frequência de um processador de acordo com
a carga de trabalho, resultando em um consumo de energia menor. \cite{elaine_et_al:14}

Essa ferramenta foi utilizada artigo proposto por possibilitar a simulação em nuvens computacionais
de workflows descritos no mesmo padrão usado pelo Pegasus, avaliando o tempo de execução e o consumo energético.Além disso, o código da ferramenta é livre, sendo assim, é possível incluir nela novos algoritmos de escalonamento de tarefas

O ambiente utilizado para este artigo é o Sistema Opecarional CentOS 6.4, com o IaaS \textit{Eucalyptus}. Mais detalhes será mostrado na seção de metodologia.
