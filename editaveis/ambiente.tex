\section{Ambiente}

No artigo analisado foi utilizado a ferramenta CloudSim e sua extensão CloudSim-DVFS. O CloudSim é
um simulador de nuvens computacionais, que pode ser utilizado para geração de uma nuvem privada.
No que se refere a nuvem privada, sua infraestrutura opera exclusivamente para uma única organização ou usuário,
seja ela gerida internamente ou por um terceiro, e hospedada internamente ou externamente.

O CloudSim-DVFS é uma extensão do simulador CloudSim que permite simulações energéticas tanto de uma
carga de trabalho quanto de workflows, além de fornecer uma nova versão da simulação da técnica
de dimensionamento dinâmico de voltagem e frequência (\textit{Dynamic voltage and frequency scaling – DVFS}).
Essa técnica permite mudar dinamicamente a voltagem e a frequência de um processador de acordo com
a carga de trabalho, resultando em um consumo de energia menor. \cite{elaine_et_al:14}

Essa ferramenta foi utilizada no artigo \cite{elaine_et_al:14} proposto por possibilitar a simulação em nuvens computacionais
de workflows descritos no mesmo padrão usado pelo Pegasus, avaliando o tempo de execução e o consumo energético.
Além disso, o código da ferramenta é livre, sendo assim, é possível incluir nela novos algoritmos de escalonamento de tarefas.

Nesse artigo foram utilizadas as ferramentas \textit{Eucalyptus}, na versão 3.1.1.1, para simulação e geração de nuvens computacionais e o \textit{VMWare}\footnote{a.k.a Virtual Machine Ware} para simulação de virtualização. O \textit{Eucalyptus}
é uma ferramente de software de código aberto para computação em nuvem que implementa o que é comumente referido
como infraestrutura como serviço (IaaS), que são sistemas que dão aos usuários a capacidade de executar e controlar
instâncias de máquinas virtuais inteiras implantados através de uma variedade de recursos físicos. \cite{nurmi_2009}

A \textit{VMWare} foi utilizada para a criação de uma máquina virtual do sistema operacional CentOS 6 para replicação do experimento
ocorrido no \cite{elaine_et_al:14}. O conceito de máquina virtual é definido pela IBM como: uma cópia do hardware físico da máquina totalmente protegido
e isolado. Assim, cada usuário de uma máquina virtual possui a ilusão de ter uma máquina física dedicada.
Os desenvolvedores de software também podem escrever e testar programas sem o medo de deixar que a máquina física
não funcione e afete outros usuários. \cite{sugerman2001virtualizing}.
