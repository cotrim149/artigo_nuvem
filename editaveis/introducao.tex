\section{Introdução}


introducao da bagaça!
