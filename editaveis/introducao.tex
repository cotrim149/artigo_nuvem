\section{Introdução}

Computação em Nuvem, também conhecida como computação sob demanda, é um paradigma da computação baseada na internet, onde os recursos compartilhados e informações são fornecidas para computadores e outros dispositivos sob demanda. A computação em nuvem fornece aos seus usuários, diferentes capacidades para armazenar e processar seus dados em databases de terceiros. Ela se baseia no compartilhamento dos recursos para atingir coerência e escalabilidade.

A computação em nuvem busca maximizar a eficácia dos recursos compartilhados, esses recursos são, em geral, não apenas compartilhados, mas também são realocados dinamicamente dependendo da demanda. Essa abordagem ajuda a maximizar o uso do poder de computação, enquanto reduz o custo total de recursos, usando menos energia para manter o sistema. A disponibilidade de redes de alta capacidade, computadores de baixo custo e dispositivos de armazenamento, bem como a adoção generalizada de virtualização de hardware e arquitetura orientada a serviços, levaram a um alto crescimento da computação em nuvem.

A principal tecnologia presente na computação em nuvem é a virtualização. O software de virtualização separa um dispositivo de computação física em um ou mais dispositivos ''virtuais'', cada um dos quais podem ser facilmente usados e gerenciados para executar tarefas de computação. Com a virtualização em nível de sistema operacional, essencialmente, é possível criar um sistema escalável de vários dispositivos de computação independentes, o que possibilita que recursos computacionais ociosos podem ser atribuídos e utilizados de forma mais eficiente. A virtualização também contribui para reduzir o custo de infra-estrutura.

Segundo \cite{elaine_et_al:14} foi realizado um experimento, que criou uma instância de aglomerado com 26.496 núcleos, usando máquinas do tipo \textit{c3.8xlarge} da Amazon EC2, foi observado que o desempenho dessa instância foi equivalente ao de uma máquina com 484,2 TeraFLOPS, o que comprova o fornecimento de um ambiente de alto desempenho gerado pela Computação em Nuvem.

O artigo está estruturado da seguinte forma:
\begin{itemize}
  \item Seção 2 será descrito o ambiente para replicação do artigo \cite{elaine_et_al:14}.
  \item Seção 3 será descrito um paralelo entre: como o artigo propôs a implementação do mesmo e como o experimento deste artigo foi realizado.
  \item Seção 4 será descrito as dificuldades, problemas e soluções em relação ao desenvolvimento da replicação dos resultados do artigo.
  \item Seção 5 será descrito os resultados.
\end{itemize}
