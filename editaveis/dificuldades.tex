\section{Dificuldades}

Este artigo foi feito baseado na atividade de refazer o experimento no artigo \cite{elaine_et_al:14} utilizando de outra tecnologia que não fosse usada no artigo alvo. O artigo alvo usa a ferramenta \textit{CloudSim}. Já o experimento de replicação está usando o \textit{Eucalyptus}. Este experimento foi feito para a matéria de Computação em Nuvem da Faculdade do Gama.

Baseado neste fato, alguns problemas relacionados a esta adaptação foram encontrados.

\begin{enumerate}

  \item Simulação de máquinas virtuais
  \begin{itemize}
    \item Problema: O Eucalyptus não faz simulação, somente a virtualização de ambiente como nuvem.
    \item Solução: Utilização de máquinas virtuais.
  \end{itemize}

  \item Simulação da quantidade requerida de máquinas virtuais
  \begin{itemize}
    \item Problema: O computador utilizado como Nó de controle\footnote{Node Controller} para o Eucalyptus, não tem poder computacional para virtualização da quantidade mínima de máquinas requerida pelo artigo alvo.
    \item Solução: Utilização de uma quantidade mínima para simulação
  \end{itemize}

  \item Instalação da ferramenta Eucalyptus
  \begin{itemize}
    \item Problema: Foi feita a escolha de instalar o Eucalyptus em uma máquina virtual com o Sistema Operacional Ubuntu 12.04 LTS, porém não existe, atualmente, as dependências que necessitam ser instaladas.
    \item Solução: Houve a troca do Sistema Operacional para CentOS 6.4 . Utilizou-se a versão \textit{fast start} para se fazer a instalação, tanto do Nó de controle, quanto do Nó de apresentação\footnote{Frontend Node}
  \end{itemize}

\end{enumerate}


\begin{comment}

  \item Instalação da ferramenta Eucalyptus
  \begin{itemize}
    \item Problema:
    \item Solução:
  \end{itemize}


\end{comment}
